\section{Héritage avancé}



\subsection{Introduction}



\subsection{Overriding de méthodes trouvées en Framework}
Le super ne doit être appelé que dans des méthode redéfinies. Et ne doit servir que à appeler la méthode de la classe parent.



\subsection{Classes abstraites}
Une classe abstraite est une classe qu’on ne peut instancier directement.
Cela permet d’implémenter des interfaces. Elle fournit une abstraction qui permettra de mettre des choses en commun entre ses filles.
Une classe abstraite l’est si elle a au moins une méthode abstraite. (\textbf{virtual} foo() = 0;).

Si la classe fille ne veut pas être abstraite elle doit redéfinir toutes les méthodes virtuelles pures (=abstraites) de la mère. Si elle le fait, on appelle ça une \textbf{concrétisation}.



\subsection{Où placer les méthodes}
Quand on ajoute une méthode à une classe, on le met le plus haut possible dans la hiérarchie des classes. L’implémentation doit aussi se baser le plus possible sur les méthodes déjà existantes.