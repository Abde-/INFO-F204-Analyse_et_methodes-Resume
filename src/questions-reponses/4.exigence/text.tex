\section{Exigence}



\subsection{Question - 3 points}



\subsubsection{Qu'est-ce qu'un SRD ?}
\textcolor[rgb]{0,0.48,0.58}{C'est un document de spécificité des besoins, il reprend les besoins d'un utilisateur pour la création d'un programme sous forme d'un rapport écrit.}



\subsubsection{À quoi sert un SRD ?}
\textcolor[rgb]{0,0.48,0.58}{Il reprend ce que les développeurs du système doivent implémenter. Il contient les exigences de l'utilisateur et du système.}



\subsubsection{À quoi faut-il faire attention lorsque l'on rédige un SRD ?}
\textcolor[rgb]{0,0.48,0.58}{A être clair et précis (éviter le langage informatique que le client ne comprendra pas…). Discuter régulièrement avec le client et tenir compte des changements. Décrire ce que fait le programme et non comment (pas de détails d'implémentation). Écrire des phrases courtes, utiliser des diagrammes pour la clarté. Inclure un index, une table des matières, utiliser la voix active, éviter les fautes d'orthographes, ...}



\subsection{Question - 2 points}



\subsubsection{Donnez la définition d’exigences non fonctionnelles (Non-functional requirements)}
\textcolor[rgb]{0,0.48,0.58}{Les exigences non concernées par les fonctions spécifiques du système.}

\subsubsection{Donnez les 3 types principaux d’exigences non fonctionnelles, avec un exemple pour chacun d’eux.}
\color[rgb]{0,0.48,0.58}
Exigences de produit (fiabilité, portabilité, efficacité, ...)
\\Exemple: fiabilité, un système de contrôle de centrale nucléaire ne doit pas planter à tout moment.

Exigences organisationnelles (implémentation, livraison, ...) 
\\Exemple: Facilité d’utilisation, un système utilisé dans des situations critiques ou par des personnes handicapées doivent être simples d’utilisation et ne pas nécessiter de manipulations précises et compliqués.

Exigences externes (sécurité, législation, ...)
\\Exemple: Interopérabilité, pour des systèmes s’intégrant dans d’autres existant déjà.
\color[rgb]{0,0,0}