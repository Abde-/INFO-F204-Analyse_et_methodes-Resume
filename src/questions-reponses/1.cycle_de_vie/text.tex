\section{Cycle de vie}



\subsection{Question - 2 points}



\subsubsection{Présentez le cycle de vie "Waterfall"}
\textcolor[rgb]{0,0.48,0.58}{On procède par phase :}
\begin{itemize}
\color[rgb]{0,0.48,0.58}
	\item \textbf{Requirements Collection}
    \\Rencontre avec le client et note de tous les besoins (Risques: Documentations incomplètes, inexactes et ambiguës)

	\item \textbf{Analysis}
    \\Les analystes définissent les besoins, les écrans, … (Risques: Fournir une spécification qui ne correspond pas aux besoins du client.)

	\item \textbf{Design Architectes}
    \\On conçoit l'architecture de l'application. Conception de diagrammes et choix des librairies.

	\item \textbf{Implementation}
    \\Les codeurs développent leurs modules.

	\item \textbf{Testing}
    \\Assemblage et livraison.
\end{itemize}
\textcolor[rgb]{0,0.48,0.58}{Cela a donné de bon résultats mais problèmes de communication car elle est faite par documents plutôt que par la parole. Les codeurs n'ont pas de recul et ne peuvent détecter des problèmes potentiels. Si il y a une erreur il faut corriger dans chacune des étapes. Un logiciel est long à développer et les besoins du client peuvent changer durant le développement.
\\Avantages : Très contrôlé, panifiable, des documents décrivant l'entièreté de l'application.}



\subsubsection{Expliquez la différence entre ce cycle de vie et le développement itératif}
\color[rgb]{0,0.48,0.58}
Le cycle Waterfall a donné de bon résultats mais problèmes de communication car elle est faite par documents plutôt que par la parole. Les codeurs n'ont pas de recul et ne peuvent détecter des problèmes potentiels. Si il y a une erreur il faut corriger dans chacune des étapes. Un logiciel est long à développer et les besoins du client peuvent changer dur le développement.
\\Avantages : Très contrôlé, panifiable, des documents décrivant l'entièreté de l'application.

Tandis que le développement itératif. On procède par incréments. On livres des morceaux de logiciel au client petit à petit. C'est un enchaînement de mini waterfall. Cette méthode permet d'avoir plus de retour du client. On ne fait pas une grosse analyse, on développe juste complètement un module qu'on va montrer au client. Il vérifie et apporte ses corrections qui seront facilement faisables. On fait cela à chaque itération. Une itération dure en moyenne entre 2 et 4 semaines.
\\Avantages : Le client s'implique et le projet à de grosses chances de réussites.
\color[rgb]{0,0,0}