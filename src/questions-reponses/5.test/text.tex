\section{Test}



\subsection{Question 3 points}



\subsubsection{Expliquez les termes suivants et leur utilité dans le contexte de la réalisation de tests unitaires}

\begin{itemize}
    \item \textbf{Pré-condition}
\end{itemize}
\textcolor[rgb]{0,0.48,0.58}{Condition booléenne qui doit être vraie avant l'appel d'une méthode.}

\begin{itemize}
	\item \textbf{Post-condition}
\end{itemize}
\textcolor[rgb]{0,0.48,0.58}{Condition booléenne qui doit être vraie après l'appel d'une méthode.}

\begin{itemize}
	\item \textbf{Invariant}
\end{itemize}
\color[rgb]{0,0.48,0.58}
Condition booléenne qui doit être vraie avant et après le traitement. Il peut temporairement être faux durant le traitement. Si un des invariant est faux, c'est qu'il y a une erreur.
\\Le test unitaire est un test où on ne teste pas l'application mais ses composants. On va créer un code supplémentaire pour réaliser cela.
\\Contrairement au test unitaire, le test exploratoire est le fait de jouer avec l'application. On utilise l'interface mais on a rien prévu pour tout tester.
\color[rgb]{0,0,0}



\subsection{Question 3 points}



\subsubsection{Expliquez en quoi consiste le test unitaire. Donnez 3 qualités que doit avoir un test unitaire. Expliquez, pour chacune de ces qualités, la raison pour la quelle cette qualité est importante.}
On va tester une classe ou un petit morceau de programme.
Il existe des frameworks de test unitaires dans presque tout les langages. ( CppUnit pour C++) En CppUnit, un testcase hérite de CppUnit::TestCase. La classe du test devrai avoir le nom “NomTest”. Un échec est une assertion qui rate, mais on l’avait anticipé. Une Erreur est un bug non-détecté par une assertion.
\\Un bon test unitaire doit avoir certaines propriétés.
\begin{itemize}
	\item Un test unitaire doit être déterministe: A chaque lancement du test, on doit avoir le même résultat. Sinon on peut perdre confiance dans les tests.
	\item Le lancement doit être automatisé, on peut les relancer très facilement.
	\item Les test sont une source de documentation. Ils doivent donc être lisible.
	\item Ils doivent être le moins sensible possible aux changements du code.
	\item Ils ne doivent pas tester des choses évidentes mais traiter les choses complexes et difficiles en premier.
\end{itemize}
